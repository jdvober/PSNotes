%NO SPACES IN NEWCOMMAND DEFINITIONS!

% For determining custom spacing under underbrackets
\newcommand*\mystrut[1]{\vrule width0pt height0pt depth#1\relax}
%Ex:
% $\underbrace{\mystrut{1.5ex}c = a + b}_{\text{my equation}}$

\newcommand{\unitdef}[2]{
	\begin{align*}
		\Aboxed{{#1} \Rightarrow {#2}}
	\end{align*}
}

\newcommand{\formuladef}[2]{
	\textbf{$\text{Solving for} \atop \text{#1}$}
	\[#2\]
	%\columnbreak % Manually breaks the column to ensure content starts in the next column
}

\newenvironment{boxFormula}[3]{
	\begin{formulaBox}{Formulas: #2} % Title
		\begin{center}
			\begin{multicols}{#1} % Specifies number of columns

				}{

			\end{multicols}
		\end{center}
	\end{formulaBox}
}

\newcommand{\showFormula}[3]{
	{\large
			\begin{multicols}{2}
				\begin{boxBackground}{When \textcolor{dracCyan}{#1} is unknown}
					#2
				\end{boxBackground}
				\begin{boxBackground}{\textcolor{dracCyan}{#1} is measured in}
					#3
				\end{boxBackground}
			\end{multicols}
		}
}

% Custom Highlighting Command
\newcommand{\highlight}[2][dracPink]{
	\colorbox{#1}{#2}
}

% For full width items in an enumerate, outline setting etc.
\newcommand{\unindent}[1]{% #1 = text
	\par\hspace*{-\@totalleftmargin}\parbox{\textwidth}{\strut #1\strut}\par}
\makeatother

% The newcommand definition
% #1: Optional named parameters
% #2: Position, e.g., (0,0) or at (0,0)
% #3: Text inside the node (not used in this specific command)
\newcommand{\gradientArrow}[2][]{
	% PGFkeys definition for the gradientArrow command
	\pgfkeys{
		/mycommands/gradientArrow/.is family,
		/mycommands/gradientArrow/.cd,
		% Define keys with their default values
		tipColor/.store in=\gradientArrowTipColor,
		tailColor/.store in=\gradientArrowTailColor,
		width/.store in=\gradientArrowWidth,
		length/.store in=\gradientArrowHeight,
		rotate/.store in=\gradientArrowRotate,
		xpos/.store in=\gradientArrowXPos,
		ypos/.store in=\gradientArrowYPos,
		textColor/.store in=\gradientArrowTextColor,
		% A style key to easily apply all defaults
		defaults/.style={
				tipColor=red,
				tailColor=green,
				width=0.25cm,
				length=4cm,
				rotate=90,
				xpos=0,
				ypos=0,
				textColor=black,
			},
	}
	\begin{scope}
		% Process the keys locally within a group
		\pgfkeys{/mycommands/gradientArrow/.cd, defaults, #1}
		\node[
			single arrow,
			minimum height=\gradientArrowHeight,
			minimum width=\gradientArrowWidth,
			draw=black,
			line width=0.5pt,
			shading=axis,
			top color={\gradientArrowTipColor},
			bottom color={\gradientArrowTailColor},
			single arrow head extend=0.2cm,
			anchor=south,
			rotate=\gradientArrowRotate
		] at (\gradientArrowXPos, \gradientArrowYPos) {\color{\gradientArrowTextColor}{#2}};
	\end{scope}
}


\newcommand{\Definition}[3]{
	\unindent{
		\begin{boxBox}{Definition}
			\textbf{\underline{{\large #1}}} - {#2} {#3}
		\end{boxBox}
	}
}

% Used when making math with braces underneath, along with text.
\newcommand{\mathBrace}[2]{
	\stackunder{#1}{\mathclap{\underbrace{\hphantom{#1}}_{#2}}}
}

% Define a custom command for the circled atoms in a chemfig diagram
\newcommand{\circledatom}[2]{%
	\tikz[baseline=(char.base)]\node[
		draw,
		circle,
		fill=#1,
		inner sep=1pt,
		minimum size=1.5em] (char) {\dracBg{#2}};
}

\newcommand{\incfig}[2][1]{%
	\def\svgwidth{#1\columnwidth}
	\import{../figures/}{#2.pdf_tex}
}
