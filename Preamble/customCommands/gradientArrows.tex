% The newcommand definition
% #1: Optional named parameters
% #2: Position, e.g., (0,0) or at (0,0)
% #3: Text inside the node (not used in this specific command)
\newcommand{\gradientArrow}[2][]{
	% PGFkeys definition for the gradientArrow command
	\pgfkeys{
		/mycommands/gradientArrow/.is family,
		/mycommands/gradientArrow/.cd,
		% Define keys with their default values
		tipColor/.store in=\gradientArrowTipColor,
		tailColor/.store in=\gradientArrowTailColor,
		width/.store in=\gradientArrowWidth,
		length/.store in=\gradientArrowHeight,
		rotate/.store in=\gradientArrowRotate,
		xpos/.store in=\gradientArrowXPos,
		ypos/.store in=\gradientArrowYPos,
		textColor/.store in=\gradientArrowTextColor,
		% A style key to easily apply all defaults
		defaults/.style={
				tipColor=red,
				tailColor=green,
				width=0.25cm,
				length=4cm,
				rotate=90,
				xpos=0,
				ypos=0,
				textColor=black,
			},
	}
	\begin{scope}
		% Process the keys locally within a group
		\pgfkeys{/mycommands/gradientArrow/.cd, defaults, #1}
		\node[
			single arrow,
			minimum height=\gradientArrowHeight,
			minimum width=\gradientArrowWidth,
			draw=black,
			line width=0.5pt,
			shading=axis,
			top color={\gradientArrowTipColor},
			bottom color={\gradientArrowTailColor},
			single arrow head extend=0.2cm,
			anchor=south,
			rotate=\gradientArrowRotate
		] at (\gradientArrowXPos, \gradientArrowYPos) {\color{\gradientArrowTextColor}{#2}};
	\end{scope}
}
