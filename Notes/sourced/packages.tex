
%%%%%%%%%%%%%%%%%%%%%%%%%%%%%%%%%%%%%%%%%%%%%%%%%%%%%%%%%%%%%%%%%%%%%%%%%%%%%%%%%%%%%%%%%%%%%%%%%%%%%%%%%%%%%%%%%%%%%%%%%%%%%%%%%%%%%%%%%%%%%%%%%%%%%%%%%%%%%
% Bibliography Package
%%%%%%%%%%%%%%%%%%%%%%%%%%%%%%%%%%%%%%%%%%%%%%%%%%%%%%%%%%%%%%%%%%%%%%%%%%%%%%%%%%%%%%%%%%%%%%%%%%%%%%%%%%%%%%%%%%%%%%%%%%%%%%%%%%%%%%%%%%%%%%%%%%%%%%%%%%%%%
% Can cause issuses with \label
% biber backend is necessary for apa style.  If this gives issues, try compiling on Overleaf instead / use latexmk to compile.
\usepackage[
	backend=biber,
	style=ieee,
]{biblatex}
\addbibresource{sources.bib}
\addbibresource{../Bibliography/sources.bib}

\usepackage{multirow} % allows combining multiple rows in tables
\usepackage{multicol}

\usepackage{lipsum} %Lorem Ipsum

\usepackage{marginnote} % For notes in the margin
\usepackage{stackengine} % Allows for stacking of math commands inside of math modes
\stackMath % Use math mode inside stacking commands

\usepackage[T1]{fontenc}
\usepackage{textcomp}
\usepackage[english]{babel}

\usepackage{csquotes} % For block quotations
\usepackage{graphicx}
\usepackage{circuitikz}
\usetikzlibrary{tikzmark, arrows.meta, positioning, shapes.arrows}
\usepackage{pgfplots}
\pgfplotsset{compat=1.8}
\usepackage{pgfkeys}
\graphicspath{ {../Resources/images/} }
\usepackage[absolute]{textpos}
\usepackage[colorlinks=true, linkcolor=dracBlue, urlcolor=dracBlue, citecolor=dracBlue]{hyperref}
\usepackage{indentfirst} % indent the first line of a new paragraph automatically
\usepackage{setspace}

\usepackage{fancyhdr}% Fancy Headers and Footers
\usepackage{lastpage}

\usepackage{outlines} % Outline environment
\usepackage{enumitem}

% Math
\usepackage{amssymb}
\usepackage{amsfonts}
\usepackage{siunitx} %https://texdoc.org/serve/siunitx/0 <----- Very important read
\usepackage{mathtools}
\usepackage[makeroom, thicklines]{cancel}

% Chemistry
\usepackage{chemfig}

\usepackage{markdown} % allows for md syntaxinside
\usepackage{pdfcomment}

% Support for appendices
\usepackage[toc,page]{appendix}

% Logic Control
\usepackage{multido}
\usepackage{forarray}
\usepackage{readarray}
\usepackage{pgffor}
