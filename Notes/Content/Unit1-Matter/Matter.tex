\documentclass[../../main.tex]{subfiles}
\graphicspath{{\subfix{./Resources/images/}}}

\begin{document}
% Main Body
\chapter{Matter}

\section{What is Matter?}
\begin{outline}[enumerate]
    \1 \colorbox{dracPink}{Matter is the “stuff” that makes up everything in the universe.}

    \begin{displayquote}
        \begin{boxBox}{Definition}
            {\large\textbf{\underline{Matter}}} - Anything that has mass and takes up space.
        \end{boxBox}
    \end{displayquote}

    \1 Properties of Matter
    \2 Each specific substance has its own combination of properties that can be used to identify the substance.
    \2 \colorbox{dracPink}{Matter can \pdftooltip{$\Delta$}{$\Delta$ means "Change"} it's properties.}
    \3 Ex. Water is a
    \4 Liquid at room temperate
    \4 Solid at cold temperatures
    \4 Gas at high temperatures
    \2 Examples:
    \3 Hardness
    \3 Texture
    \3 Flammability
    \3 Color
    \3 Shape
    \3 Temperature

    \begin{displayquote}
        \begin{boxBox}{Definition}
            {\large\textbf{\underline{Chemistry}}} - the study of the properties of substances and how matter changes.
        \end{boxBox}
    \end{displayquote}

    \section{Kinds of Matter}

    \1 3 Kinds

    % Elements
    %     |     combine together to get
    %     V
    % Compounds
    %     |     mix together to get
    %     V
    % Mixtures
    \begin{tikzpicture}[>=latex, colorbar arrow/.style={
                    shape=double arrow,
                    double arrow head extend=0.125cm,
                    shape border rotate=90,
                    minimum height=5cm,
                    shading=#1
                }]
        % Box styles
        \tikzstyle{rect}=[draw=black,
        rectangle,
        fill=gray,
        fill opacity=0.2,
        text opacity=1,
        minimum width=75pt,
        minimum height=25pt,
        align=center]

        % Nodes for boxes
        \node[rect] (elems) {Elements};
        \node[rect, above=of elems] (cmpds) {Compounds};
        \node[rect, above=of cmpds] (mixs) {Mixtures};

        % Arrows between boxes
        \draw[->, thick] (elems.north) -- (cmpds.south) node[midway, xshift=20mm]{\red{Combine} together to get};
        \draw[->, thick] (cmpds.north) -- (mixs.south) node[midway, xshift=16mm]{\red{Mix} together to get};

        \draw[->] (cmpds.north)--(mixs.south)node[midway, xshift=16mm]{\red{Mix} together to get};
        % Define the Starting and Ending RGB Colors
        % Start Color (Bottom): RGB(189, 147, 249) -> A purplish color
        % End Color (Top):    RGB(255, 121, 198) -> A pinkish color

        % 1. Define the Colormap (Gradient)
        \pgfplotsset{
            colormap={newgradient}{
                    % Start Color (at position 0)
                    rgb255=(189,147,249)
                    % End Color (at position 1)
                    rgb255=(255,30,50)
                }
        }

        % 2. Draw the Gradient-Filled Arrow
        \node[
            single arrow,
            minimum height=4cm,
            minimum width=0.25cm,
            draw=black,
            line width=0.5pt,
            anchor=south,
            shading=axis,
            % Crucial: Update the colors to use the RGB values
            % The gradient starts at the bottom and ends at the top.
            top color={rgb:red,255; green,30; blue,50},    % The color at the top (light pinkish)
            bottom color={rgb:red,189; green,147; blue,249}, % The color at the bottom (purplish)
            single arrow head extend=0.2cm,
            rotate=90              % Rotates the arrow 90 degrees to point UPWARDS
        ] at (-2,1.75) {\dracFg{Complexity}};
    \end{tikzpicture}

    \1 Elements

    \begin{displayquote}
        \begin{boxBox}{Definition}
            {\large\textbf{\underline{Element}}} - A substance that is made up of only one type of atom. \autocite{SimpleWiki-ChemicalElement}
        \end{boxBox}
    \end{displayquote}

    \2 If you break down an element any more, then it just becomes generic \emph{protons}, \emph{neutrons} and \emph{electrons}.
    \3 It stops behaving like that element
    \\- Ex: If you break down Gold into protons, neutrons and electrons, it is no longer a shiny metal that conducts electricity.
    \2

    \1 Compounds

    \begin{displayquote}
        \begin{boxBox}{Definition}
            {\large\textbf{\underline{Compound}}} - A chemical compound is a substance made of two or more different elements joined together by chemical bonds in a fixed ratio.
            \autocite{SimpleWiki-ChemicalCompound}
        \end{boxBox}
    \end{displayquote}

    \1 Mixtures

\end{outline}

\newpage
\begin{formulaBox}{Density}
    \showFormula{Density}{\center$density = \frac{mass}{volume}$}{\center\unitsDensity}
    \color{dracPurple}\rule{\textwidth}{1.5pt}\color{black}
    \showFormula{Mass}{\center$mass = density \cdot volume$}{\center\unitsMass}
    \color{dracPurple}\rule{\textwidth}{1.5pt}\color{black}
    \showFormula{Volume}{\center$volume = \frac{mass}{density}$}{\center\unitsVolume}
\end{formulaBox}

\printbibliography
\end{document}
