\documentclass[../../main.tex]{subfiles}
\graphicspath{{\subfix{./Resources/images/}}}

\begin{document}
% Main Body
\chapter{Matter}

\section{What is Matter?}
\begin{outline}[enumerate]
    \1 \colorbox{dracPink}{Matter is the “stuff” that makes up everything in the universe.}

    \Definition{Matter}{Anything that has mass and takes up space.}{\autocite{a2003_matter}}

    \1 Properties of Matter
    \2 Each specific substance has its own combination of properties that can be used to identify the substance.
    \2 \highlight{Matter can \pdftooltip{\marginnote{$\Delta$ means "Change"}{$\Delta$}}{$\Delta$ means "Change"} it's properties.}
    \3 Ex. Water is a
    \4 Liquid at room temperate
    \4 Solid at cold temperatures
    \4 Gas at high temperatures
    \2 Examples:
    \3 Hardness
    \3 Texture
    \3 Flammability
    \3 Color
    \3 Shape
    \3 Temperature

    \newpage

    \Definition{Chemistry}{The science that studies what everything is made of and how it changes.}{\autocite{a2003_chemistry}}

    \section{Kinds of Matter}

    % Elements
%     |     combine together to get
%     V
% Compounds
%     |     mix together to get
%     V
% Mixtures
\begin{tikzpicture}[>=latex, colorbar arrow/.style={
				shape=double arrow,
				double arrow head extend=0.125cm,
				shape border rotate=90,
				minimum height=5cm,
				shading=#1
			}]
	% Define the 'rect' style for nodes
	\tikzset{rect/.style={draw, rectangle, rounded corners, minimum width=2cm, minimum height=1cm, fill=white}}

	% Nodes for boxes
	\node[rect] (elems) {Elements};
	\node[rect, above=of elems] (cmpds) {Compounds};
	\node[rect, above=of cmpds] (mixs) {Mixtures};

	% Arrows between boxes
	\draw[->, thick] (elems.north) -- (cmpds.south) node[midway, xshift=20mm]{\red{Combine} together to get};
	\draw[->, thick] (cmpds.north) -- (mixs.south) node[midway, xshift=16mm]{\red{Mix} together to get};

	\draw[->] (cmpds.north)--(mixs.south)node[midway, xshift=16mm]{\red{Mix} together to get};
	Define the Starting and Ending RGB Colors
	Start Color (Bottom): RGB(189, 147, 249) -> A purplish color
	End Color (Top):    RGB(255, 121, 198) -> A pinkish color

	\gradientArrow[xpos=-1.5cm, ypos=1.8cm, length=5cm, tailColor=dracGreen, tipColor=dracRed]{Complexity}

\end{tikzpicture}


    \1 Elements

    \Definition{Element}{A substance that is made up of only one type of atom.}{\autocite{a2003_chemical}}

    \2 If you break down an element any more, then it just becomes generic \emph{protons}, \emph{neutrons} and \emph{electrons}.
    \3 It stops behaving like that element
    \begin{itemize}
        \item Ex: If you break down Gold into protons, neutrons and electrons, it is no longer a shiny metal that conducts electricity.
    \end{itemize}
    \2 Each element has its own symbol
    \3 Usually the first 1 - 2 letters in the name
    \3 \highlight{Always CAPITAL\ lowercase if two letters long}
    \3 Examples
    \begin{itemize}
        \item \highlight{O $\rightarrow$ \underline{O}xygen}
        \item \highlight{He $\rightarrow$ \underline{He}lium}
        \item C $\rightarrow$ \underline{C}arbon
        \item H $\rightarrow$ \underline{H}ydrogen
        \item Al $\rightarrow$ \underline{Al}uminum
        \item Au $\rightarrow$ \marginnote{The latin word for Gold is "\underline{Au}rum", so it still follows the rule, just in a different language.}{Gold}
    \end{itemize}


    \1 Compounds

    \Definition{Compound}{A chemical compound is a substance made of two or more different elements joined together by chemical bonds in a fixed ratio.}{\autocite{a2004_chemical}}


    \2 Ex: Carbon Dioxide ($CO_2$)\\

    \vspace{1cm}

    \chemfig{%
        \circledatom{dracOrange}{O}-\circledatom{dracBg}{\dracFg{C}}-\circledatom{dracOrange}{O}
    }

    \vspace{1cm}
    \2 Ex: Water ($H_{2}O$)\\

    \chemfig{%
        \circledatom{dracOrange}{O}%
        (-[:-52.25]\circledatom{dracCyan}{H})%
        (-[:232.25]\circledatom{dracCyan}{H})% 180+ 52.5 <-- lower left hydrogen
    }

    \Definition{Chemical Formula}{A combination of symbols that show the ratio of elements in a compound.}{\autocite{a2006_chemical}}

    \2 Examples
    \- \(CO_2\)

    \1 Mixtures

\end{outline}

\newpage
\begin{formulaBox}{Density}
    \showFormula{Density}{\center$density = \frac{mass}{volume}$}{\center\unitsDensity}
    \color{dracPurple}\rule{\textwidth}{1.5pt}\color{black}
    \showFormula{Mass}{\center$mass = density \cdot volume$}{\center\unitsMass}
    \color{dracPurple}\rule{\textwidth}{1.5pt}\color{black}
    \showFormula{Volume}{\center$volume = \frac{mass}{density}$}{\center\unitsVolume}
\end{formulaBox}

\end{document}
