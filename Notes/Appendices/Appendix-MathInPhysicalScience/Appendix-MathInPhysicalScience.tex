\documentclass[../../main.tex]{subfiles}
\graphicspath{{\subfix{./Resources/images/}}}

\begin{document}
% Main Body
\chapter{Math in Physical Science}

\section{Numbers}

Generally, in this class, you should give your final answers in decimal form.  If it is a fraction that makes sense, like \(\frac{4}{5}\), fine, give it as a fraction.  If it ends up being some weird fraction like \(\frac{85}{217}\), please just use decimal form.\\

\begin{boxPink}{Rounding Decimals}
	Answers should be rounded to two or three decimal places.
\end{boxPink}

\section{Units}

\section{Examples of solving Math Problems in Physical Science}

Now that we have discussed what units are in Math and Science, let's look the general steps you should be doing for all of the math problems in this class.  We will use an example problem from near the beginning of the year involving Gas Laws.

\begin{boxGray}{Example 1}
	A balloon is filled with air.  The pressure of the balloon is \orange{\(10~atm\)} to start.  This expands it to a starting volume of \blue{\(2~mL\)}.  The balloon is then squeezed to a new pressure of \orange{\(28~atm\)}.  What would be the \blue{new volume} of the balloon after it is squeezed?
\end{boxGray}

When solving math-based problems in this class, you will \emph{generally} follow the following five steps:

\begin{enumerate}
	\item Write down the numbers that you are \textbf{GIVEN} (with units).
	\item Write down the number you are asked to \textbf{FIND} (with units.)
	      \begin{enumerate}
		      \item I usually write a \textbf{\purple{?}} instead of a number, because we are going to solve for this.  We don't know what it is yet!
	      \end{enumerate}
	\item Narrow down your formulas until you have one that allows you to solve for the missing number.
	      \begin{enumerate}
		      \item Look at your list of \textbf{GIVENS} and \textbf{FINDS} and pick a formula that has the same variables in it as are in the list
		            \begin{enumerate}
			            \item For instance, if you are given Pressures and Volumes, and are asked to find a Volume, you should look for a formula involving Pressures and Volumes
			                  \begin{enumerate}
				                  \item This would be Boyle's Law
			                  \end{enumerate}
		            \end{enumerate}
	      \end{enumerate}
	\item Get the correct version of the formula you chose, so that it reads \[\textbf{\emph{FIND}} = ...\text{other stuff}\]
	      \begin{enumerate}
		      \item You can do this with algebra (if you have learned that in your math class)

		            OR

		      \item Pick the correct "version" of the formula from the list provided by Mr. Vober.  Check the wall or your notes for the different "versions".
	      \end{enumerate}
\end{enumerate}

\end{document}
