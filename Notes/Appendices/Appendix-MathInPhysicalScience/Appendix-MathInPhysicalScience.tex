\documentclass[../../main.tex]{subfiles}
\graphicspath{{\subfix{./Resources/images/}}}

\begin{document}
% Main Body
\chapter{Math in Physical Science}

\section{Numbers}

Generally, in this class, you should give your final answers in decimal form.  If it is a fraction that makes sense, like \(\frac{4}{5}\), fine, give it as a fraction.  If it ends up being some weird fraction like \(\frac{85}{217}\), please just use decimal form.\\

\begin{boxPink}{Rounding Decimals}
	Answers should be rounded to two or three decimal places.
\end{boxPink}

\section{Units}

\section{Examples of solving Math Problems in Physical Science}

Now that we have discussed what units are in Math and Science, let's look the general steps you should be doing for all of the math problems in this class.  We will use an example problem from near the beginning of the year involving Gas Laws.

\begin{boxGray}{Example 1}
	A balloon is filled with air.  The pressure of the balloon is \orange{\(10~atm\)} to start.  This expands it to a starting volume of \blue{\(2~mL\)}.  The balloon is then squeezed to a new pressure of \orange{\(28~atm\)}.  What would be the \blue{new volume} of the balloon after it is squeezed?
\end{boxGray}

\end{document}
