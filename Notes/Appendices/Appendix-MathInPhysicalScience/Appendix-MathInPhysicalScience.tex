\documentclass[../../main.tex]{subfiles}
\graphicspath{{\subfix{./Resources/images/}}}

\begin{document}
% Main Body

\chapter{Math in Physical Science}

\section{Numbers}

Generally, in this class, you should give your final answers in decimal form.  If it is a fraction that makes sense, like \(\frac{4}{5}\), fine, give it as a fraction.  If it ends up being some weird fraction like \(\frac{85}{217}\), please just use decimal form.\\

\begin{boxPink}{Rounding Decimals}
	Answers should be rounded to two or three decimal places.
\end{boxPink}

\section{Scientific Notation}

	\subsection{Why}

		Scientific Notation is just a more convienent way to display really big and really small numbers.  For instance, instead of writing

		\begin{quote}
			{\huge 6,020,000,000,000,000,000,000,000}
		\end{quote}

		over and over in a problem, it is much more convienent to write

		\begin{quote}
			{\huge \num{6.02e23}}
		\end{quote}

	\newpage
	\subsection{How to read and write Scientific Notation}

\begin{figure}[!ht]
	\begin{circuitikz}
		\tikzstyle{every node}=[font=\LARGE]
		%Words on top
		\node [font=\LARGE] at (1,14.75) {Decimal};
		\node [font=\LARGE] at (4,14.75) {Base};
		\node [font=\LARGE] at (7,14.75) {Exponent};

		%Text on bottom
		\node [font=\LARGE] at (4,12) {$\num{6.02}~~\text{x}10^{~~23}$};

		%Rectangles
		\draw  (2.25,12.5) rectangle (3.65,11.5);
		\draw  (3.775,12.5) rectangle (4.85,11.5);
		\draw  (4.975,12.5) rectangle (5.65,11.75);

		%Arrows
		\draw [->, >=Stealth] (1,14.25) -- (2.75,12.75);
		\draw [->, >=Stealth] (4.125,14.25) -- (4.25,12.75);
		\draw [->, >=Stealth] (7,14.25) -- (5.5,12.75);
	\end{circuitikz}
	\label{fig:my_label}
\end{figure}


The \emph{Decimal} portion of the number is just all of the numbers expressed so you have
\begin{quote}
	\quad\quad\quad One digit \textbf{.} (All the other numbers)
\end{quote}

Examples:
\bigskip
1.157 \quad 2.23 \quad 1.050433404

The \emph{Base} portion of the number is either \num{e25}\sisetup{output-exponent-marker=\ensuremath{\mathrm{E}}} or \num{e25}
\dracComment{\marginnote{Calculators with one line displays often show exponents as \num{e25} instead of \sisetup{output-exponent-marker=} x\num{e25} }} to save space.  Either is fine by me.

The exponent tells you how far you have to move the decimal place to get back to the "normal" way of writing numbers\dots

If you have a positive exponent, it means you move the decimal to the right.  If you have a negative exponent, it means you move the decimal to the left.\\

Negative exponents $\rightarrow$ numbers < 1.

Positive exponents $\rightarrow$ big numbers at least > 1

\section{Units}\label{sec:AppendixUnits}

\quote{Units of measurement give standards so that the numbers from our measurements refer to the same thing.} \autocite{SimpleWikiUnits}

Pretend someone tells you that they have a new cousin.  You ask how old they are.  They say 5.  You might automatically assume that they mean \num{5}~\unit{\years} old.  What if they just got a new baby cousin, who is \qty{5}{months\ old}?

Telling someone HOW something was measured is pretty important.  We call the way we measured something a Unit.

\begin{displayquote}
	\begin{boxBox}{Unit}
	\textcolor{dracRed}{{\large\textbf{\underline{Unit}}}} - What is used to tell HOW something is measured.  This comes after the number. Examples: \qty{15}{\textbf{seconds}} or
	\end{boxBox}
\end{displayquote}

\section{Examples of solving Math Problems in Physical Science}

Now that we have discussed what units are in Math and Science, let's look the general steps you should be doing for all of the math problems in this class.  We will use an example problem from near the beginning of the year involving Gas Laws.


\subsection{General Steps}
When solving math-based problems in this class, you will \emph{generally} follow the following five steps:

 \begin{enumerate}
	\item Write down the numbers that you are \textbf{GIVEN} (with units).
	\item Write down the number you are asked to \textbf{FIND} (with units.)
	      \begin{enumerate}
		      \item I usually write a {\huge\textbf{\purple{?}}} instead of a number, because we are going to solve for this.  We don't know what it is yet!
	      \end{enumerate}
	\item Narrow down your formulas until you have one that allows you to solve for the missing number.
	      \begin{enumerate}
		      \item Look at your list of \textbf{GIVENS} and \textbf{FINDS} and pick a formula that has the same variables in it as are in the list
		            \begin{enumerate}
			            \item For instance, if you are given Pressures and Volumes, and are asked to find a Volume, you should look for a formula involving Pressures and Volumes
			                  \begin{enumerate}
				                  \item This would be Boyle's Law
			                  \end{enumerate}
		            \end{enumerate}
	      \end{enumerate}
	\item Get the correct version of the formula you chose, so that it reads \[\textbf{\emph{FIND}} = ...\text{all the other variables you are given}\]
	      \begin{enumerate}
		      \item You can do this with algebra (if you have learned that in your math class)

		            OR

		      \item Pick the correct "version" of the formula from the list provided by Mr. Vober.  Check the wall or your notes for the different "versions".
	      \end{enumerate}
	\item Plug in the \textbf{GIVENS} into the matching places in the formula you chose.
	\item Solve the math problem
	\item Record you final answer, \textbf{with units}.
	\begin{enumerate}
		\item The units for the answer will be the same as the same type of number in the \textbf{GIVENS}
		\item Example: You are solving for a mass, and the \textbf{GIVEN} mass is measured in $kg$ $\rightarrow$ your answer will also be in $kg$
	\end{enumerate}
\end{enumerate}

\newpage
\subsection{Worked Example 1}

\begin{boxGray}{Example 1}
	A balloon is filled with air.  The pressure of the balloon is \orange{\qty{10}{atm}} to start.  This expands it to a starting volume of \blue{\qty{2}{\milli\liter}}.  The balloon is then squeezed to a new pressure of \orange{\qty{28}{atm}}.  What would be the \blue{new volume} of the balloon after it is squeezed?
\end{boxGray}

\subsubsection{Find}

\marginnote{\tikzmarknode{find1_start}"What would be the \blue{new volume} of the balloon after it is squeezed?" is asking us to find $V_2$}[-2.5cm]

$\orange{V_2} = \orange{?}$\tikzmark{find1_end}

\begin{tikzpicture}[overlay, remember picture]
    \draw [->] ([yshift=-1.5cm, xshift=-0.25cm]pic cs:find1_start) -- ([xshift=8.25cm, yshift=0.5cm]pic cs:find1_end);
\end{tikzpicture}

\subsubsection{Given}

$\blue{P_1} = \blue{\qty{10}{atm}}$
\quad\quad
$\orange{V_1} = \orange{\qty{2}{\milli\liter}}$
\quad\quad
$\blue{P_2} = \blue{\qty{28}{atm}}$\tikzmark{given1_end}
\marginnote{\tikzmarknode{given1_start}The problem tells a story.  I can tell if the variable is $P$, $V$ or $T$ from the units.  The $_1$ and $_2$ are from reading the story.  If the value is from \textit{before} we changed something, then it is a $_1$, otherwise, it is from \textit{after} we changed something, and it gets a $_2$}[-1.5cm]

\begin{tikzpicture}[overlay, remember picture]
    \draw [->] ([yshift=-1.5cm, xshift=-0.25cm]pic cs:given1_start) -- ([xshift=2.25cm, yshift=0.125cm]pic cs:given1_end);
\end{tikzpicture}

\subsubsection{Formula}

\[\textcolor{Maroon}{\underbrace{\mystrut{6ex}\textcolor{orange}{V_2}}_{\text{The "Find" variable} \atop \text{by itself on the left}}} \textcolor{black}{=} \textcolor{Maroon}{\underbrace{\textcolor{black}{\frac{\blue{P_1} \cdot \orange{V_1}}{\mystrut{4ex}\blue{P_2}}}}_{\text{All the variables} \atop \text{you are "Given"}}}\]\tikzmark{formula1_end} % use \atop to split lines in underbraces.
\marginnote{\tikzmarknode{formula1_start}I pick a formula that has the thing we are asked to find on one side, and all of the other variables on the other.}[1cm]

\begin{tikzpicture}[overlay, remember picture]
    \draw [->] ([xshift=-0.25cm]pic cs:formula1_start) -- ([xshift=9.75cm, yshift=0.5cm]pic cs:formula1_end);
\end{tikzpicture}


\subsubsection{Work}
\begin{align*}
\orange{V_2} &= \frac{\blue{P_1} \cdot \orange{V_1}}{\blue{P_2}}\\
\\
\orange{V_2} &= \frac{(\blue{\qty{10}{\Ccancel[red]{atm}}}) \cdot (\orange{\qty{2}{\milli\liter}})}{(\blue{\qty{28}{\Ccancel[red]{atm}}})}\marginnote{You can treat the units and numbers almost as two seperate things.  See \ref{sec:AppendixUnits} for more details.}
\\
\orange{V_2} &= \frac{(\orange{\qty{20}{\milli\liter}})}{(\qty{28})}\\
\end{align*}

\subsubsection{Answer with units}
\begin{align*}
\Aboxed{\orange{V_2} &= \orange{\qty{0.71}{\milli\liter}}}
\end{align*}

\marginnote{0.71428571428 rounded to two decimal places.}

\end{document}
