\documentclass[../../main.tex]{subfiles}
\graphicspath{{\subfix{./Resources/images/}}}

\begin{document}
% Main Body
\chapter{Taking Excellent Notes}

\section{The old, terrible way of taking notes}

Most of your notes look something like this; just a wall of text:\\

\begin{boxRed}{How most freshmen take notes}
Matter is anything that has mass and takes up space (volume). It's the stuff that makes up everything we can see and touch, from the smallest atom to the largest galaxy. We can classify matter in a few ways. First, we can look at its physical state. The three main states are solid, liquid, and gas. A solid has a definite shape and volume; its particles are packed tightly together and vibrate in place. Think of a block of ice or a rock. A liquid has a definite volume but no definite shape, taking the shape of its container. Its particles are close but can slide past one another. Water is a great example. A gas has no definite shape or volume, and its particles are far apart and move randomly and quickly. The air we breathe is a mixture of gases like nitrogen and oxygen. There's also a fourth state, plasma, which is a super-heated gas where atoms are stripped of their electrons. It's found in stars and lightning. Beyond states, we can also classify matter as a pure substance or a mixture. A pure substance has a fixed composition and consistent properties throughout, like gold or distilled water. A mixture, on the other hand, is a combination of two or more substances that are not chemically bonded and can be separated by physical means. Think of a salad or salt water.  Matter is also classified into pure substances: elements and compounds. An element is the simplest form of matter and cannot be broken down into a simpler substance by chemical means. Every element is made up of only one type of atom. The periodic table is a complete list of all the known elements, such as carbon (C), oxygen (O), and iron (Fe).
\end{boxRed}

This is VERY hard to use later.  You can't find anything when you need it, and you need to do a TON of reading.\\

There is a better way.

\section{The EASY way to get great notes}

\begin{outline}[enumerate]

\1 Excellent notes use an \underline{Outline} format
	\2	\dracRed{This is required for my class to get your points for notes!}
	\2 This works in all of your classes.
\1 Examples
	\begin{boxGreen}{Good}
	\marginnote{Note the good indentation $\mapsto$}

	1. Main Idea 1

	\quad a. Detail 1

	\quad b. Detail 2

	\marginnote{Indentation $\mapsto$ represents more specific stuff}

	\quad \quad i) Detail about Detail 2

	\quad \quad i) Another detail about Detail 2

	\quad c. Detail 3

	2. Main Idea 2

	\quad \quad etc...
	\end{boxGreen}

	\begin{boxRed}{Bad}
	\marginnote{Nothing is indented.  This is hard to read and find information later.  Indenting is an easy way to make your notes better.}

	1. Main Idea 1

	a. Detail 1

	b. Detail 2

	i) Detail about Detail 2

	i) Another detail about Detail 2

	c. Detail 3

	2. Main Idea 2

	\quad \quad etc...

	\end{boxRed}


	\1 It is better to over-indent than under-indent.

	\1 Style

		\2 Choose whatever style you like the most.  You can use any combination of the following:

	\begin{displayquote}

	1. Numbers

	\quad a. Letters

	\quad \quad i) Roman Numerals

	\quad \(\bullet\) Bullet Points

	\quad \quad $\Box$ Boxes

	\quad $\rotatebox[origin=c]{180}{$\Lsh$}$ Curly Arrows

	\quad \quad - Dashes
	\end{displayquote}


	\1 Other useful symbols and conventions
	\marginnote{Put misc. things you want to remember in the margins.}

	$\bullet$ \quad \textbf{Bolding}, \underline{Underlining} and \underline{\underline{Double Underlining}} your text to represent important words.

	$\bullet$ \quad $\Delta$ Greek letter "Delta".  In math and science, means "Change".

	$\bullet$ \quad $\Rightarrow$ Double arrows for definitions.

	$\bullet$ \quad $\leadsto$ Squiggly arrows for saying when one things leads to another thing.

	$\bullet$ \quad $\approx$ For when things are about the same.

	$\bullet$ \quad $\dashrightarrow$ Dashed arrows.

	$\bullet$ \quad $\bigstar$ For really important stuff that you want to call out.

	$\bullet$ \quad \fbox{Put a box around definitions.}

	$\bullet$ \quad \fbox{\fbox{Double Box formulas.}}

	\1 Fancy Box Ideas

	\begin{boxBluePointy}
		Blue Pointy Box
	\end{boxBluePointy}

	\begin{boxBoxTitle}{Title}
		Centered Title Box
	\end{boxBoxTitle}

	\begin{boxcompare}{Step By Step Math}
		\begin{equation}
		i = \frac{n(n+1)}{2}.
		\end{equation}
		\tcblower
		\begin{equation}
		\sum\limits_{i=1}^n i = \frac{n(n+1)}{2}.
		\end{equation}
	\end{boxcompare}



	\1 Advanced / Extra stuff

	\2 \dracPurple{To really take your notes to the next level, incorporate colors.}

	\2 \colorbox{dracCyan}{\dracRed{Bring different colored pens and highlighters to draw attention to specific details.}}

	\margincloud{Put questions or thoughts in the margins with clouds}

	\2 \dracOrange{Come up with your own system of what each color means.}

	\1 Abbreviations

	\2 To take notes QUICKLY and keep up, you must abbreviate (a.k.a. shorten) as much as possible, while still having what you write down make sense later.

\end{outline}
\end{document}
